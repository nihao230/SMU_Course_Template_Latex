%!TeX program = xelatex
\documentclass[12pt,hyperref,a4paper,UTF8]{ctexart}
\usepackage{SMUReport}

%% --- 请在此处粘贴上面的【代码高亮设置】 ---
%% 如果你已经把设置放入了 SMUReport.sty,则忽略此步
\usepackage{listings}
\usepackage{xcolor}
\definecolor{codegreen}{rgb}{0,0.6,0}
\definecolor{codegray}{rgb}{0.5,0.5,0.5}
\definecolor{codepurple}{rgb}{0.58,0,0.82}
\definecolor{backcolour}{rgb}{0.95,0.95,0.92}
\lstdefinestyle{mystyle}{
    backgroundcolor=\color{backcolour},   
    commentstyle=\color{codegreen},
    keywordstyle=\color{magenta},
    numberstyle=\tiny\color{codegray},
    stringstyle=\color{codepurple},
    basicstyle=\ttfamily\footnotesize,
    breakatwhitespace=false,         
    breaklines=true,                 
    captionpos=b,                    
    keepspaces=true,                 
    numbers=left,                    
    numbersep=5pt,                  
    showspaces=false,                
    showstringspaces=false,
    showtabs=false,                  
    tabsize=2
}
\lstset{style=mystyle}
%% ---------------------------------------

%%-------------------------------正文开始---------------------------%%
\begin{document}

%%-----------------------封面--------------------%%
%%-----------------------封面--------------------%%
\cover
\newpage

%%------------------摘要-------------%%
% 1. 摘要和目录页通常归为前言部分,使用罗马数字页码
\pagenumbering{Roman} 
\setcounter{page}{1}

\begin{abstract}
    % 将摘要加入目录(可选)
    \addcontentsline{toc}{section}{摘要} 
    本文档旨在作为一份使用指南...(此处省略)...
\end{abstract}

% 此时页脚是 'i'。如果你想让摘要页完全不显示页码,可以用 \thispagestyle{empty}
% 但注意,下一页(目录)会变成 'ii'。如果想让目录从 'i' 开始,需要在目录处再次重置。

%%--------------------------目录页------------------------%%
\newpage
\tableofcontents

%%------------------------正文页从这里开始-------------------%
\newpage

%================ 设置第一页为GITHUB REPO ================
\fancypagestyle{firstpage}{%
  % 1. 先保留页眉设置 (把 SMUReport.sty 里的页眉代码复制过来)
  \fancyhead[L]{\footnotesize \fangsong 上海海事大学模板}
  \fancyhead[R]{\footnotesize \fangsong \leftmark}
  
  % 2. 恢复页眉的横线 
  \renewcommand{\headrulewidth}{0.5pt} 
  
  % 3. 设置页脚
  \fancyfoot[C]{\thepage} % 中间保留页码
  \fancyfoot[R]{ GitHub: \url{https://github.com/nihao230/SMU_Course_Template_Latex}} % 右侧加链接
  % 注意:如果链接太长,建议用 \tiny 缩小字体,防止和页码重叠
}

% ... (在正文第一页使用)
\thispagestyle{firstpage}

% 2. 【关键修正】这里必须切换回阿拉伯数字,否则正文会全是罗马数字
\pagenumbering{arabic} 
\setcounter{page}{1} 

% 应用自定义的第一页页脚(包含Github链接)
% 注意:\thispagestyle 必须放在 \pagenumbering 之后,否则可能失效
\thispagestyle{firstpage} 

\section{模版使用说明}
本模板主要适用于课程平时论文及期末论文。默认设置如下:
\begin{itemize}
    \item 页边距:2.5cm
    \item 字体:中文宋体,英文 Times New Roman
    \item 字号:12pt(小四)
\end{itemize}

\subsection{文件结构}
\begin{table}[!htbp]
    \centering
    \begin{tabular}{l | l}
    \hline
        文件名 & 说明 \\
        \hline
        \texttt{main.tex}  & 主编译文件(即当前文件) \\
        \texttt{reference.bib} & 参考文献数据文件 \\
        \texttt{SMUReport.sty}  & 样式控制文件(页眉、标题、代码设置等)\\
        \texttt{figures/}  & 图片存放文件夹 \\
        \hline
    \end{tabular}
    \caption{文件组成说明}
\end{table}

\section{代码插入功能 (多语言支持)}
本模版已集成 \texttt{listings} 宏包,支持语法高亮。使用 \verb|\begin{lstlisting}[language=...]| 即可。

\subsection{C++ 代码示例}
引用 C++ 代码时,设置 \texttt{language=C++}:

\begin{lstlisting}[language=C++, caption={C++ Hello World 示例}, label={code:cpp}]
#include <iostream>
#include <vector>

// 这是一个C++注释
int main() {
    std::vector<std::string> msg = {"Hello", "SMU"};
    for (const auto& word : msg) {
        std::cout << word << " ";
    }
    std::cout << std::endl;
    return 0;
}
\end{lstlisting}

\subsection{Python 代码示例}
引用 Python 代码时,设置 \texttt{language=Python}:

\begin{lstlisting}[language=Python, caption={Python 列表推导式}, label={code:py}]
import numpy as np

def calculate_area(r):
    """ 计算圆的面积 """
    return np.pi * r**2

# 生成列表
radius_list = [1, 2, 3, 4, 5]
areas = [calculate_area(r) for r in radius_list]
print(f"Areas: {areas}")
\end{lstlisting}

\subsection{Matlab 代码示例}
引用 Matlab 代码时,设置 \texttt{language=Matlab}:

\begin{lstlisting}[language=Matlab, caption={Matlab 绘图脚本}, label={code:matlab}]
% 清除工作区
clc; clear; close all;

t = 0:0.01:2*pi;
y = sin(t);

figure(1);
plot(t, y, 'LineWidth', 2);
title('Sine Wave');
xlabel('Time (s)');
ylabel('Amplitude');
grid on;
\end{lstlisting}

\section{页码设置教程}
很多课程对页码的起始位置有不同要求,以下是两种最常见的配置方法,请根据需求在 \texttt{main.tex} 中调整。

\subsection{方案一:页码从正文第一页开始(当前默认)}
这种方案下,目录页通常使用罗马数字(i, ii),或者不标页码,正文第一页标记为第 1 页。

\textbf{实现代码:}
\begin{lstlisting}[language=TeX]
% --- 在目录页之前 ---
\newpage
\pagenumbering{Roman} % 目录页使用罗马数字
\tableofcontents

% --- 在正文第一章之前 ---
\newpage
\pagenumbering{arabic} % 切换回阿拉伯数字
\setcounter{page}{1}   % 强制将当前页设置为第1页
\section{第一章}
\end{lstlisting}

\subsection{方案二:页码从目录页开始计数}
这种方案下,目录页即为第 1 页,后续正文顺延(例如目录是 1-2 页,正文从第 3 页开始)。

\textbf{实现代码:}
\begin{lstlisting}[language=TeX]
% --- 在目录页之前 ---
\newpage
\pagenumbering{arabic} % 直接使用阿拉伯数字
\setcounter{page}{1}   % 目录页设为第1页
\tableofcontents

% --- 在正文第一章之前 ---
\newpage
% 此时不需要任何设置,页码会自动顺延
\section{第一章}
\end{lstlisting}

\section{其他常用功能}

\subsection{文本框}
使用 \verb|\tbox| 命令插入重点提示:
\tbox{
    这是一个圆角灰底的文本框,适合放置结论、定义或重要提示。
}

\subsection{公式与图表}
行内公式示例:$E = mc^2$。

行间公式示例:
\begin{equation}
    J(\theta) = \frac{1}{2m} \sum_{i=1}^{m} (h_\theta(x^{(i)}) - y^{(i)})^2
\end{equation}

校徽图片引用如 \autoref{SJTU} 所示。

\begin{figure}[!htbp]
    \centering
    % 请确保 figures 目录下有该图片,或者替换为你自己的图片
    \includegraphics[width =.3\textwidth]{figures/smu_logo.pdf} 
    \caption{上海海事大学校徽}
    \label{SMU}
\end{figure}

\section{参考文献引用}
使用 \verb|\cite{}| 进行引用。例如:
本文参考了关于对抗攻击的研究\cite{ExampleRef1},以及相关的深度学习理论\cite{ExampleRef2}。

%%----------- 参考文献 -------------------%%
\reference

\end{document}